\selectlanguage{english}%

Given the plethora of data and services today available online, it is often difficult to find on-the-fly the information or the applications that are appropriate to the current context of use. This is even more accentuated in the mobile scenario, where device resources (memory, computational power, transmission budget) are still limited. 

Given this evidence, this thesis focuses on the definition of methods and tools for the design and development of \emph{Context-Aware Mobile mashUpS} (\emph{CAMUS}). \emph{CAMUS} apps dynamically collect and integrate data from documental, social and Web resources (accessed by means of Web APIs) and adapt the integrated contents to the users’ situational needs. They can offer multiple advantages thanks to their intrinsic capability of identifying pertinent data sources, selected on the basis of their adequateness with respect to the current users’ needs, and pervasively presenting them to the final user in the form of integrated visualizations deployed as mobile apps.

This application paradigm overcomes the limits posed by pre-packaged apps and offers to users flexible and personalized applications whose structure and content may even emerge at runtime based on the actual user needs and situation of use. An example of benefit is the integration with support services, such as maps, weather forecast or public transport information, to provide a better user experience.

This thesis presents a design method and an accompanying platform for the development of \emph{CAMUS} app. The approach is characterized by the role given to context as a first-class modelling dimension used to support \emph{i)} the identification of the most adequate resources that can satisfy the users' situational needs and \emph{ii)} the consequent tailoring at runtime of the provided data and functions. Context-based abstractions are exploited to generate models specifying how data returned by selected services have to be merged and visualized by means of integrated views. These models then drive the flexible execution of the final mobile app on target mobile devices. A prototype of the platform, making use of novel and advanced web and mobile technologies, is also illustrated.

\selectlanguage{italian}%
