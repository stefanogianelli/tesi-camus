Data la sovrabbondanza di dati e servizi disponibili online è spesso difficile trovare informazioni e applicazioni adatte al contesto corrente. Questa situazione è accentuata se la ricerca di dati è eseguita su dispositivi mobili, dove le risorse (memoria, capacità computazionale, piano dati modesto) sono limitate.

Date queste premesse questa tesi mira alla definizione di metodologie e strumenti per la progettazione e sviluppo di \emph{Context-Aware Mobile mashUpS} (\emph{CAMUS}). Le applicazioni \emph{CAMUS} recuperano e integrano i dati dinamicamente da risorse online (interrogate tramite web \emph{API}) e si occupano di ritagliare le informazioni acquisite in base alla situazione nella quale si trova l'utente. Offrono numerosi vantaggi grazie alla capacità di identificare fonti pertinenti, selezionate in base alla loro adeguatezza rispetto alle esigenze dell'utente, i cui dati sono integrati e forniti sotto forma di \emph{mobile app}.

Questo paradigma permette di superare i limiti delle applicazioni preconfezionate e mette a disposizione applicazioni flessibili e personalizzate la cui struttura e contenuti possono variare in fase di esecuzione in base alla situazione d'uso. Un esempio di beneficio è l'integrazione dei contenuti principali con servizi di supporto, come le mappe, le informazioni sul meteo o il trasporto pubblico, che possono migliorare l'esperienza d'uso.

Questa tesi presenta una metodologia di progettazione e una piattaforma per la creazione di applicazioni CAMUS. L'approccio utilizza la modellazione del contesto come strumento principale per \emph{i)} individuare le risorse più adatte per soddisfare i bisogni dell'utente e \emph{ii)} individua in fase di esecuzione come ritagliare i dati e le funzioni recuperate dalle risorse selezionate. L'astrazione fornita dal contesto, infatti, viene utilizzata per generare modelli che definiscono come i dati recuperati andranno integrati e visualizzati. Questi modelli definiscono le regole di esecuzione dinamica delle \emph{mobile app}. La tesi inoltre illustra un prototipo che fa uso delle tecnologie \emph{web} e \emph{mobile} più recenti per supportare la modellazione delle applicazioni CAMUS, la generazione automatica del codice e l'esecuzione delle applicazioni finali.