Internet sta diventando una fonte sempre più importante per l'acquisizione di informazioni. Cambiano anche i metodi di interazione: in passato Internet veniva consultato maggiormente da PC mentre ai giorni d'oggi si tende ad accederci tramite smartphone o tablet. Questi dispositivi hanno avuto un progresso tecnologico estremamente rapido, diventando sempre più sofisticati. Tutto questo progresso non porta solo benefici: siamo infatti inondati da una enorme quantità di informazioni, che rendono difficoltoso il processo di ricerca di ciò che è realmente utile.

CAMUS si inserisce proprio in questo contesto: vuole sfruttare le potenzialità messe a disposizione dai dispositivi mobili per fornire informazioni di reale interesse per chi le deve esplorare. Per fare questo si andrà a sfruttare il \emph{contesto} nel quale si trova l'utente, dove per contesto si intende l'insieme di informazioni sulla situazione nella quale si trova l'utente e il proprio profilo personale, cioè le sue preferenze e interessi. Questi dati raccolti verranno sfruttati per cercare dalle fonti più idonee e per filtrare i risultati che sono effettivamente utili nella situazione corrente.

Un altro importante aspetto riguarda la modalità di presentazione di queste informazioni. Si vuole creare un sistema che dia flessibilità nella creazione delle interfacce grafiche attraverso cui mostrare i dati. Questi schemi verranno sfruttati anche per fornire dei dati aggiuntivi relativi alle informazioni cercate, come mappe, informazioni sul meteo e sui trasporti. Si ritiene che l'utilizzo di queste informazioni di supporto possa migliorare la qualità dei dati forniti e completi in modo ottimale le informazioni di base ottenute grazie al contesto.