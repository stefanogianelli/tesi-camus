Data la sovrabbondanza di dati e servizi disponibili online è spesso difficile trovare rapidamente le informazioni e applicazioni adatte al contesto corrente. Questa situazione viene accentuata nel caso vengano effettuate ricerche da dispositivi mobili, dove le risorse (memoria, capacità computazionale, piano dati limitato) sono limitate.

Questa tesi mira alla definizione di metodologie e strumenti per la progettazione e sviluppo di \emph{Context Aware mashUpS} (\emph{CAMUS}). Le applicazioni \emph{CAMUS} recuperano e integrano i dati dinamicamente da documenti, \emph{social network} e risorse online (interrogate tramite web \emph{API}) e si occupano di adattare le informazioni acquisite in base alla situazione nella quale si trova l'utente. Offrono numerosi vantaggi grazie alla capacità di identificare fonti pertinenti, selezionate in base alla loro adeguatezza rispetto alle esigenze dell'utente, e le mostrano integrate all'utente sotto forma di \emph{mobile app}.

Questo paradigma permette di superare i limiti delle applicazioni preconfezionate e mette a disposizione dell'utente applicazioni flessibili e personalizzate dove la struttura e i contenuti possono variare in fase di esecuzione in base alle esigenze dell'utente e alla situazione nella quale si trova. Un esempio di beneficio è l'integrazione con i servizi di supporto, come le mappe, le informazioni sul meteo o il trasporto pubblico, che migliorano l'esperienza d'uso.

Questa tesi presenta una metodologia di progettazione e una piattaforma per la creazione di applicazioni CAMUS. L'approccio è basato dal ruolo del contesto come strumento principale per il supporto \emph{i)} dell'identificazione delle risorse più adatte per soddisfare i bisogni dell'utente e \emph{ii)} la preparazione in fase di esecuzione dei dati e delle funzioni recuperate. L'astrazione fornita dal contesto viene utilizzata per generare modelli che definiscono come i dati recuperati andranno integrati e visualizzati. Questi modelli definiscono le regole di esecuzione dinamica delle \emph{mobile app}. Viene inoltre illustrato un prototipo della piattaforma, che fa uso delle tecnologie \emph{web} e \emph{mobile} recenti e sofisticate.