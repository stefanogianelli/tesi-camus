\textcolor{red}{sistemare ...}
L'applicazione mobile CAMUS rappresenta l'altro elemento fondamentale dell'ar\-chi\-tet\-tu\-ra. Fornisce all'utente finale l'interfaccia per accedere al sistema e semplifica l'utilizzo e l'accesso ai dati. Il capitolo si apre con la presentazione dell'architettura interna dell'applicazione e le motivazioni che hanno spinto verso una soluzione di questo tipo. Viene introdotto lo schema per definire le pagine e la logica di aggiornamento dei dati, seguendo il principio del flusso dati unidirezionale. In seguito è definita la struttura del file di mashup proveniente dal server che è indispensabile per la creazione delle schermate, per poi trattare nello specifico i metodi che occorrono per costruire queste schermate.
Successivamente si tratterà del flusso dei dati all'interno dell'applicazione e di come vengono gestite le richieste dati verso il server, tramite GraphQL, per effettuare le richieste riguardanti il login e i risultati contestuali per l'utente. In conclusione verrà trattato l'argomento riguardante i servizi di supporto.

\section{Struttura dello schema di mashup\label{sec:struttura-schemi-mashup}}

Per poter permettere una dinamicità nella visualizzazione delle schermate che formano l'applicazione si è scelto di utilizzare uno schema di mashup molto semplice e allo stesso tempo in grado di fornire i parametri necessari all'applicazione per costruire le viste in tempo reale. Si è scelto di utilizzare una rappresentazione in JSON in modo da essere facilmente interpretabile all'interno del motore JavaScript di React Native. 
Nel caso di CAMUS sono considerate due tipologie di schema di \emph{mashup}: la \emph{list} e la \emph{details}. 
%cambiare questa frase 
Nella tipologia \emph{list} sono definite le view che definiscono il singolo item della ListView e generalmente si compone di una quantità non eccessiva di componenti. 
Nel caso della \emph{list} sono definiti gli elementi da mostrare quando è presente la lista complessiva di risultati nella \emph{ListView}.
Nella tipologia \emph{details} sono disponibili tutti i dettagli dell'item selezionato e sono definiti tutti i termini necessari a identificare l'elemento per l'utente.
Considerando l'esempio dei ristoranti, per la lista sono associati i termini che servono a identificare l'oggetto, come il nome e l'indirizzo. Se poi l'utente vuole vedere la mappa e visualizzare gli estremi per contattare il ristorante, deve aprire la pagina dei dettagli, dove acquisendo i termini dal file di mashup, sono visualizzati tutti i dettagli necessari per l'utente. Inoltre è specificata la tipologia dell'elemento da visualizzare, in modo da permettere all'app di utilizzare le librerie di collegamento con le applicazioni specifiche del sistema operativo.

Nel Listato \ref{lst:mashup-schema} viene mostrato un frammento del file JSON di mashup della pagina di dettaglio:

\begin{listing}[H]
	\inputminted{json}{6-implementazione-app/Codice/mashup-schema-example-app.json}
	\caption{Esempio Schema Mashup}
	\label{lst:mashup-schema}
\end{listing}

Di seguito sono elencate le diverse tipologie di dato che compongono l'oggetto che rappresenta i file di \emph{mashup}:

\begin{itemize}
	\item \textbf{Type}
	Il termine \emph{type} indica il tipo dell'elemento da visualizzare nella view. È associato a un componente React Native, che verrà richiamato nella fase di costruzione dell'interfaccia grafica
	\item \textbf{Topic}
	Si tratta di un array di stringhe che indica tutte le aree di interesse associate allo schema in questione. Si è scelto di utilizzare una struttura basata su tag: una volta che l’utente ha scelto l’Interest Topic viene utilizzato lo schema che lo possiede nei suoi tag 
	\item \textbf{Contents}
	Indicano i termini del mashup o dei nuovi componenti innestati. Se nell'oggetto padre è specificato un tipo gli elementi in \emph{contents} rappresentano i termini da richiedere nella query GraphQL. In caso contrario questo campo indica che in \emph{contents} è presente un nuovo livello di annidamento. Ci possono essere tanti annidamenti, fino a quando non si arriva ad avere un oggetto che possiede il campo \emph{type}
	\item \textbf{Style}
	Questo campo serve per definire uno stile da assegnare al componente. Permette di personalizzare l’aspetto del componente, andando a ridefinire lo stile predefinito. Viene utilizzato Flexbox per la specifica degli stili
\end{itemize}

\section{Rendering delle view}\label{sec:rendering-view}

Si tratta della funzionalità più delicata da progettare all'interno dell'applicazione, perché non è così semplice pensare di generare in modo dinamico delle pagine con così tanta mutabilità a partire da un documento strutturato delle interfacce visuali.
Mentre in codice nativo questo problema può presentare delle difficoltà a livello implementativo, l'utilizzo da parte di React di un Virtual DOM simile al linguaggio HTML offre il vantaggio di poter generare molto rapidamente una traduzione dal file di mashup a un modello interpretabile dal motore di rendering dell'app.
Per questo motivo è stato creato un componente apposito, il \emph{View Builder}, che, dato un oggetto contenente il lo schema di \emph{mashup}, riesce a gestire l'associazione dei contenuti semantici all'interno di questo con i componenti presenti nell'applicazione, restituendo il componente finito.
Sono presenti all'interno due tipologie di rendering: il rendering del contesto e il rendering dei risultati.

\subsection{Rendering a partire dal contesto}

Nel caso del rendering della pagina di contesto non si ritiene necessario l'utilizzo di un mashup esterno, in quanto tutte le informazioni presenti all'interno del CDT sono sufficienti a definire in modo corretto la sua visualizzazione e la sua compilazione da parte dell'utente. %cambiare frase precedente
Per ogni oggetto facente parte del CDT viene eseguita un'associazione a componenti già esistenti della view che non sono modificabili dall'esterno.

\begin{itemize}
	\item \textbf{Interest Topic}
	Gli \emph{Interest Topic} hanno bisogno di essere selezionati nella pagina iniziale appena dopo il login e caricati in una lista di bottoni con due elementi per riga, con un'icona rappresentativa. I bottoni e le icone sono predefinite nell'applicazione e necessitano soltanto di una associazione \emph<nome, elemento> dal campo \emph{Interest Topic} del CDT. Questa funzione molto semplice viene eseguita direttamente all'interno della pagina principale senza la necessità di utilizzare il View Render
	\item \textbf{Altri elementi}
	Gli altri elementi del contesto sono elaborati all'interno del componente View Builder. Con una funzione ricorsiva viene scandito l'intero CDT ricevuto dal server con i campi da modificare e in base ai contenuti viene costruito l'elemento della view, a seconda della tipologia. Per esempio, se per un campo è richiesto l'inserimento di un numero verrà mostrato un TextInput che permette il completamento solo con numeri, mentre se ho a disposizione una scelta tra più elementi verrà mostrata una lista selezionabile.
	Assume una grande importanza anche la gestione delle esclusioni tra le operazioni. Se viene selezionato \virgolette{With Car} non è corretto mostrare le possibili tipologie di trasporto pubblico, perché si conosce già a priori che l'utente non selezionerà mai nello stesso momento come mezzo di trasporto l'automobile e il bus. Per questo scopo all'interno della \emph{Context Store} vengono gestiti con una struttura dati ad Array i nodi che non devono essere mostrati e questa variabile viene aggiornata con le selezioni che sono state fatte nell'interfaccia utente all'interno della pagina di contesto. Si prenda come esempio la selezione del mezzo di trasporto: se l'utente seleziona \virgolette{WithCar}, non è necessario mostrargli la possibilità di scegliere le tipologie di mezzi di trasporto pubblico, mentre queste opzioni gli sono mostrate solo nel caso in cui selezioni l'opzione \virgolette{Public Transport}
\end{itemize}
	
\subsection{Rendering dei Risultati}\label{sec:view-risultati}

Per quanto concerne i risultati ottenuti dalle query GraphQL entrano in gioco in maniera preponderante gli schemi di mashup. Per i motivi spiegati in \ref{sec:mashup-design}, la risoluzione del problema necessita di un funzionamento più complesso rispetto alla visualizzazione del contesto. La funzione di rendering per quanto riguarda i risultati è la medesima sia per la lista che per i dettagli, quello che cambia è il contenitore del componente \emph{View Builder}: per la lista è un figlio nel metodo di \emph{renderItem()} della ListView, mentre per i dettagli definisce l'intera pagina.
Al componente View Render sono necessari due oggetti: il file di mashup e i dati del singolo risultato.
Si suppone che nel file di mashup gli elementi siano già stati generati in ordine di visualizzazione, perché nella funzione di costruzione  delle view non è possibile modificare tale ordine. Tuttavia è possibile modificare lo stile più esterno del componente risultante per cambiare il layout dei figli del \emph{View Builder}. Per prima cosa il \emph{View Builder} deve scegliere le view associate all'\emph{Interest Topic} facendo una ricerca della prima vista che contiene il valore corrente dell'\emph{Interest Topic} e passare questo oggetto alla funzione di rendering. A questo punto si è pronti per costruire la view del componente: viene scandito ogni elemento del file di mashup e si associa il valore dato dal termine dell'oggetto del risultato e viene restituito il componente pronto per essere elaborato per la visualizzazione. Nel caso in cui non fosse presente un valore per il termine desiderato ovviamente non viene visualizzato alcun elemento, si restituisce una \emph{View} base vuota, che non ha nessuna incidenza sulla grafica avendo dimensione nulla. Nel caso in cui esista un oggetto di tipo \emph{style}, esso ha la priorità rispetto a quello presente nel foglio di stile predefinito dell'applicazione. Per il momento sono disponibili queste tipologie di componenti specificabili nel campo \emph{type} dell'oggetto di mashup che sono pienamente supportati:

\begin{itemize}
	\item \textbf{Text}
	\upe il componente di testo base; per la sua differenziazione si utilizzano attributi diversi nel foglio di stile in comune per tutta l'applicazione o le specifiche presenti nell'attributo \emph{style} del mashup
	\item \textbf{Map}
	\upe il componente che visualizza le mappe. Presenta due implementazioni differenti per quanto riguarda iOS e Android, poiché come mappe di sistema utilizzano due applicazioni differenti. Per iOS visualizzate presenti le mappe di Apple, già integrate nel componente \emph{MapView} fornito da React Native, mentre per Android vengono utilizzate le mappe di Google Maps, per le quali è stato utilizzato un modulo aggiuntivo
	\item \textbf{Website}
	Viene utilizzato per gli indirizzi web. Permette di aprire il collegamento direttamente nel browser del dispositivo
	\item \textbf{Email}
	Permette di visualizzare l'indirizzo mail associato all'elemento, selezionando l'icona di invio email l'utente viene reindirizzato all'applicazione o alla scelta di un'applicazione per comporre un messaggio da inviare all'indirizzo specificato
	\item \textbf{Phone}
	Questo componente gestisce il numero di telefono dell'elemento, con la possibilità di inviare un messaggio al numero specifico o di effettuare una telefonata, sempre utilizzando le applicazioni predefinite del dispositivo mobile
	\item \textbf{Support}
	Si tratta del componente che gestisce i servizi di supporto con i collegamenti alle applicazioni esterne, verrà spiegato nei dettagli nella Sezione \ref{sec:servizi-supporto-app}
\end{itemize}

\subsection{Gestione degli stili}

Per quanto riguarda la gestione degli stili si è scelto di adottare una soluzione molto semplice, ma allo stesso tempo molto efficace. Si è partiti dalla condizione che React Native utilizza come gestione dello stile Flexbox\footnote{Flexbox su W3Schools: \url{Htp://www.w3schools.com/css/css3_flexbox.asp}}, che è una variante basata sugli stili CSS ottimizzata per la gestione degli elementi che necessitano di adattarsi a differenti dimensioni degli schermi. Il linguaggio utilizzato risulta essere molto simile a JSON e quindi è facilmente integrabile in JavaScript. L'elemento che gestisce gli attributi grafici è lo \emph{StyleSheet}, che espone l'oggetto \emph{styles} e gli attributi \emph{PRIMARY\_COLOR} e \emph{SECODARY\_COLOR}.
Nell'oggetto \emph{styles} sono impostati tutti gli stili di ogni componente grafico dell'applicazione che vengono richiamati nella fase di costruzione delle view. In questo modo ogni modifica all'interno dell'oggetto si ripercuote in tutta l'applicazione.
I due attributi per la gestione dei colori sono risultati indispensabili in particolare per alcuni componenti, come la \emph{StatusBar} di Android, che hanno come elemento impostabile solo il colore e non lo stile completo. Ovviamente la modifica all'interno dello \emph{StyleSheet} dei parametri di colore è necessaria solo in un punto del codice, in modo da propagarsi in modo coerente per l'intera applicazione.
Nel Listato \ref{lst:stylesheet} viene mostrato un frammento del foglio di stile utilizzato nell'applicazione. Si noti come i parametri che definiscono ogni singolo elemento sono abbastanza autoesplicativi, fornendo allo sviluppatore una maggiore semplicità di programmazione. Diversi parametri sono stati ripetuti, per permettere la visualizzazione del componente in modo ottimale sia in Android che in iOS.

\begin{listing}[H]
	\inputminted{js}{6-implementazione-app/Codice/stylesheet.js}
	\caption{Frammento Foglio di Stile}
	\label{lst:stylesheet}
\end{listing}

\section{Query GraphQL}\label{sec:utilizzo-dati-app}

I dati nell'intero sistema CAMUS assumono un ruolo fondamentale. In questa sezione si vuole affrontare la gestione dei dati all'interno dell'applicazione, partendo dagli algoritmi per costruire le query GraphQL per poi parlare dell'utilizzo dei dati all'interno dell'applicazione, tramite la gestione della paginazione.
Tutto lo scambio di dati tra l'applicazione e il server viene svolto utilizzando query GraphQL, come spiegato nella Sezione \ref{sec:graphql-introduzione}. Per implementare il modulo di gestione delle connessioni si è scelto di utilizzare \emph{Lokka}\footnote{Lokka: \url{https://github.com/kadirahq/lokka}}, un client compatibile con GraphQL che permette di gestire le query come se fossero normali \emph{fetch()} di JavaScript e lasciare una configurazione più libera e adattabile alle esigenze del sistema. Questo si sposa bene con il paradigma Flux, salvando la maggior parte del contenuto della risposta nella \emph{Data Store} con le modalità esposte nella Sezione \ref{sec:action-store} e rendendolo quindi disponibile per tutti i singoli componenti dell'applicazione. Di seguito sono spiegate le due tipologie principali di query gestite con GraphQL, che sono le query di login e le query per accedere ai dati.

\subsection{Login Query}

Le query per gestire il meccanismo di login sono molto semplici. Sono basate sui dati inseriti dall'utente quando si trova nella \emph{Login Page} e sulle risposte intermedie fornite dal server. Queste query servono per ottenere l'identificativo del CDT associato all'utente e lo schema dei mashup con le definizioni delle sue view.
I dati necessari per la costruzione della query sono l'indirizzo email e la password, che sono inseriti dall'utente nella prima pagina che gli viene mostrata. A questo punto questi dati vengono salvati nella \emph{User Store} e sono tenuti nello stato dell'applicazione, nel caso in cui fosse necessario effettuare una nuova richiesta richiederli nuovamente. Nel caso in cui l'indirizzo email e/o la password siano errati viene notificato un errore mentre se uno dei due campi è mancante vengono richiesti nuovamente senza inviare nessuna richiesta.
In caso di risposta affermativa viene restituito all'utente un \emph{token}, generato ogni volta che l'utente effettua l'autenticazione. Questo \emph{token} viene salvato sempre nella \emph{User Store} e successivamente inviato al server con la query \emph{getPersonalData()} per ottenere il CDT completo per poter costruire il contesto da mostrare all'utente e lo schema di mashup.

L'indirizzo mail e il token sono inviati al server per ottenere il CDT e gli schemi di \emph{mashup} associati all'utente.

\subsection{Data Query} \label{sec:data-query}

Le query per gestire i dati necessitano di tutta una serie di operazioni preliminari perché sono presenti diversi parametri da ricavare per poter formare una richiesta GraphQL. Oltre alla definizione spiegata nella Sezione \ref{sec:endpoint-graphql}, di seguito è spiegata la modalità di costruzione dei parametri che sono necessari per la composizione della query utilizzando il metodo \emph{client.query()} di \emph{Lokka}:

\begin{itemize}
	\item \textbf{Email}
	Si tratta della mail dell'utente estratta dal database del server
	\item \textbf{IdCdt}
	Rappresenta l'identificativo del CDT. Come spiegato nelle sezioni precedenti questo identificativo appartiene a ogni singolo utente
	\item \textbf{Context}
	\upe l'oggetto che rappresenta il contesto attuale nel quale si trova l'utente. Quando l'utente ha terminato la compilazione del contesto nella \emph{Context Selection Page} i parametri e i valori sono registrati nella Context Store, ma per essere interpretati dal server necessitano di una conversione in un oggetto ben definito. Il problema di una query GraphQL è che l'oggetto in questione deve essere ulteriormente convertito in una stringa unica per poi venire collocato come payload nella POST/HTTP. Per svolgere questo compito viene utilizzata una funzione personalizzata che permette di mantenere inalterata la chiave degli oggetti JavaScript e produce in uscita una stringa. Per poter convertire anche gli oggetti che sono innestati a un livello superiore viene utilizzata una ricorsione che richiama la prima funzione
	\item \textbf{Support}
	Rappresenta la tipologia di servizi di supporto che l'utente richiede per integrare al meglio i dati che gli necessitano. %Per esempio nella demo di CAMUS si utilizzano i servizi di tipo \virgolette{Transport}
	\item \textbf{PrimaryResults}
	Rappresentano i dati che devono essere ritrovati dai servizi invocati dal server per costruire l'insieme dei risultati per l'utente. GraphQL permette di eseguire delle richieste di dati altamente personalizzate, permettendo di richiedere la minima quantità di dati indispensabile per l'utilizzo da parte dell'utente. Per poter svolgere questo compito si è scelto di utilizzare i termini semantici dei mashup, in particolare della tipologia \emph{details}, perché si suppone che nel caso della lista si abbia un sottoinsieme dei termini semantici. Come per la funzione di costruzione della view espressa nella Sezione \ref{sec:rendering-view}, anche in questo caso è necessario acquisire i mashup associati all'\emph{Interest Topic} e poi selezionare i termini, che verranno posizionati in una stringa e inviati al server.
	\item \textbf{SupportResults}
	Lo stesso meccanismo utilizzato per i \emph{PrimaryResults} viene riproposto anche per i risultati di supporto. Invece di recuperare i mashup dalle \emph{Store} viene riproposta la stessa definizione per ogni query, perché i servizi di supporto sono definiti sempre nello stesso modo con i campi \emph{category}, \emph{service} e \emph{url}.
\end{itemize}

\subsection{Gestione paginazione}\label{sec:paginazione-app}

Per quanto riguarda la gestione dei risultati l'applicazione sfrutta la paginazione messa a disposizione da GraphQL. A livello client viene implementata quando viene restituito il risultato della query basata sul contesto, per velocizzare e ottimizzare lo scambio dati col server. Nel modulo \emph{Connection Manager} sono implementati due metodi diversi per creare le query, uno per la prima query e un altro per le successive. Si tratta di due richieste non molto differenti tra loro:

\begin{enumerate}
	\item \textbf{Richiesta iniziale}
	Nella richiesta iniziale è necessario acquisire i parametri del contesto costruiti dalle \emph{Context Selection Page} e gli oggetti ricavati dai parametri definiti dall'utente e dai sensori, sotto forma in stringhe. Inoltre vengono definiti i parametri sulle dimensioni della pagina, ma senza nessun riferimento agli elementi precedenti.
	I parametri dell'oggetto ritornato dal server che contiene le informazioni necessarie all'applicazione per richiedere le pagine successive è specificato nella Sezione \ref{sec:execute-query-endpoint}
	\item \textbf{Richiesta pagine successive}
	Quando vengono richieste le pagine successive è sbagliato far ricavare nuovamente dai sensori e dall'utente il contesto, perché non è detto che questo rimanga statico. Per esempio se l'utente è in viaggio in treno le coordinate geografiche cambiano molto velocemente e quindi per il server si tratta di un nuovo contesto, che implica la restituzione di un nuovo insieme di risultati. In questo modo si potrebbero ottenere dal server dei dati diversi. Per risolvere questo problema si è scelto di memorizzare nella \emph{Context Store} i parametri immutabili della prima richiesta, come quelli di contesto, le definizioni dei dati e i servizi di supporto. La nuova query quindi recupera l'identificativo dell'ultimo elemento restituito dalla query della pagina precedente, ponendo quest'ultimo come parametro \emph{after} e mantenendo costante la dimensione della pagina
\end{enumerate} 

Questi due metodi sono chiamati dall'applicazione in due fasi differenti: la richiesta iniziale viene invocata quando l'utente ha appena definito il contesto e richiede nuovi dati, la richiesta delle pagine successive quando l'utente si trova nella \emph{Results Page} e scorre fino in fondo la lista di risultati. 
Il server riesce a gestire lo stato delle richieste come spiegato nella Sezione \ref{sec:descrittore-paginazione} e l'applicazione deve stare attenta solamente a controllare il campo che indica l'esistenza o meno di una pagina successiva.
Considerando sempre il Listato \ref{lst:esempio-risposta-data-app}, è mostrata una risposta a una query iniziale con numero di risultati nel parametro \emph{first} impostato a \emph{5}, e con il parametro \emph{hasNextPage} uguale a \emph{true} per indicare che esistono ulteriori pagine da acquisire. Nel caso in cui l'utente voglia visualizzare altri risultati è necessario eseguire una richiesta di una nuova pagina. Nella query per la seconda pagina è necessario aggiungere il parametro \emph{after} seguito dall'identificativo dell'ultimo elemento della risposta precedente assieme al parametro \emph{first} con lo stesso valore della richiesta precedente. A questo punto, essendoci già presenti dei risultati nella \emph{Data Store}, i nuovi risultati vengono concatenati assieme a quelli esistenti e aggiornano la view posizionandoli in coda a quelli già mostrati. Come ulteriore ottimizzazione nell'implementazione della \emph{ListView} è possibile impostare un'azione nel recuperare delle viewquando si giunge a una certa distanza prestabilita dalla fine della pagina. Quindi la richiesta della pagina successiva viene eseguita poco prima di raggiungere il termine della lista, fornendo all'utente un'esperienza d'uso migliore.

\section{Servizi di supporto}\label{sec:servizi-supporto-app}

Come spiegato nella Sezione \ref{sec:architettura-sistema}, i servizi di supporto sono utilizzati esclusivamente nell'applicazione. Quando viene effettuata una richiesta viene inserita la tipologia desiderata di servizi di supporto e il server restituisce un indirizzo che l'applicazione può gestire in diversi modi:

\begin{itemize}
	\item \textbf{Web Linking}
	L'applicazione riceve un link per accedere all'endpoint del servizio, richiede i dati e li visualizza all'interno del componente preposto dell'app. Questa soluzione è gestita mediante una funzione del \emph{Connection Manager} che gestisce le query nella modalità richiesta dal servizio da interrogare
	\item \textbf{Deep Linking}
	I \emph{Deep Link} sono degli URI che permettono di lanciare una applicazione preinstallata sul dispositivo. Sono collegamenti molto simili agli URL HTTP, con la differenza che viene specificato il nome che identifica l'applicazione da lanciare. Se l'applicazione desiderata non è presente sul dispositivo è necessario gestire l'errore e reindirizzare l'utente sull'App Store o sul Google Play Store per poterla scaricare. Questa tipologia di collegamenti migliora l'integrazione dell'applicazione CAMUS con l'intero sistema operativo. 
\end{itemize}
	 
In generale il funzionamento di queste tipologie di collegamenti è gestito dall'ap\-pli\-ca\-zio\-ne con le stesse modalità. Nel risultato proveniente dal server viene restituito un link del tipo \emph{nomeapp://?param={value}} e l'applicazione è in grado di sostituire al posto di \emph{value} il valore proveniente dall'elemento dei risultati per poter aprire l'applicazione di supporto. La scelta che è stata fatta è di utilizzare due link diversi per ogni sistema operativo: il primo punta all'applicazione installata sul dispositivo, mentre il secondo allo store predefinito per scaricare l'applicazione. Questa scelta permette di definire applicazioni diverse o di usare la stessa applicazione nel caso in cui utilizzi link diversi a seconda del sistema operativo. Un esempio di applicazione che ha questo problema è Google Maps: in Android l'URI è del tipo \emph{google.navigation:q={value}}, mentre in iOS è del tipo \emph{comgooglemaps://?saddr={value}}. In React Native la libreria di \emph{Linking} è in comune tra le diverse piattaforme e quindi è necessario prestare attenzione al collegamento che viene passato. Nell'applicazione questo viene implementato con il metodo \emph{Linking.canOpenUrl(URI)} nel quale viene passato al suo interno il link da aprire. Se il metodo ritorna un errore allora viene aperto il collegamento che punta all'App Store mentre in caso di risposta affermativa l'applicazione desiderata viene aperta. Nel Listato \ref{lst:esempio-support-app} viene rappresentato un esempio completo dell'oggetto che rappresenta il servizio di supporto proveniente dal server, dove:

\begin{itemize}
	\item \textbf{Category}
	Indica la categoria di appartenenza del servizio
	\item \textbf{Service}
	Rappresenta il nome del servizio
	\item \textbf{Url}
	Indica l'Url per invocare il servizio o aprire l'applicazione
	\item \textbf{Store Link}
	Rappresenta il link nello store di sistema per scaricare l'ap\-pli\-ca\-zio\-ne se non è installata sul dispositivo  
\end{itemize}

\begin{listing}[H]
	\inputminted{json}{6-implementazione-app/Codice/esempio_supporto_app.json}
	\caption{Esempio di servizi di supporto con intent}
	\label{lst:esempio-support-app}
\end{listing}