\section{Overview su cross platform mobile}
Al momento di scegliere come implementare l'applicazione mobile per CAMUS sono state considerate due opzioni: la creazione di un'applicazione nativa inizialmente solo per Android, per poi estendere la compatibilit� su iOS, oppure l'utilizzo di strumenti cross platform, funzionanti su pi� di un sistema operativo mobile.
Questa esigenza � nata dal fatto che il mercato delle app mobile sta crescendo in maniera esponenziale e le competenze richieste agli sviluppatori sono molto variegate per sviluppare applicazioni native.
Per esempio per quanto riguarda lo sviluppo iOS � necessario conoscere come linguaggi di programmazioni Objective-C e Swift, per Android ci si basa principalmente su Java e per Windows Phone � necessario utilizzare C-sharp \textcolor{red}{(Non mi prende il cancelletto)}, senza dimenticare altri sistemi operativi meno diffusi o emergenti (Tizen, Ubuntu Touch, ecc.).
Gli strumenti di sviluppo \textbf{cross-platform} solitamente si basano sul principio di \virgolette{write once run everywhere}, cio� il codice viene scritto una volta sola per creare applicazioni per diverse piattaforme, permettendo quindi di limitare le competenze richieste ai programmatori. 
\upe possibile identificare due famiglie di strumenti per la programmazione multi piattaforma:
\begin{enumerate}
	\item \textbf{Rendering tramite componenti nativi} In questa famiglia il rendering dell'applicazione viene fatto utilizzando le API grafiche native del sistema operativo di destinazione.
	\textcolor{red}{allungare + esempi}
	\item \textbf{Rendering tramite WebView} In questa famiglia il rendering dell'applicazione � svolto in una pagina web, che viene caricata delle volte in un container composto da componenti nativi.
	In questo caso � possibile programmare l'applicazione come se fosse una pagina web, implementandolo allo stesso modo di un sito web per browser.
	Purtroppo questo nel \virgolette{look and feel} dell'applicazione non � sempre un vantaggio, perch�, nonostante le possibilit� di scelta grafiche e di comportamento siano molteplici, le prestazioni possono risentire del fatto che si stia interagendo con una pagina web e non con una applicazione nativa.
\end{enumerate}

 


\section{Tecnologie utilizzate}
Per la creazione di CAMUS si � scelto di utilizzare anche lato backend degli strumenti che siano implementabili in maniera semplice e dove la suddivisione in moduli singoli riutilizzabili assuma un ruolo di primo piano. Per questo motivo si � scelto di utilizzare degli strumenti tecnologici che sono all'avanguardia nello svolgere questo compito, come React e la sua derivazione per la programmazione mobile React Native. Successivamente � introdotta la parte logica dell'applicazione con il funzionamento architetturale dell'aggiornamento dati, con il paradigma Flux.

\subsection{React}

React nasce come libreria open-source rilasciata nel 2013 da Facebook, che permette di ottimizzare le visualizzazioni delle pagine HTML, utilizzando componentine racchiudono altri specificati come HTML tag personalizzati.
Essa proviene da XHP, che � un framework HTML per il linguaggio PHP, ed � stata prima utilizzata nel newsfeed di Facebook nel 2011 e pi� tardi in Instagram.com.
Le principali funzionalit� in React sono le seguenti:
\begin{itemize}
	\item \textbf{One-way data flow} Le propriet�, un set di valori immutabili, sono passati al componente figlio all'interno del suo tag HTML.Il componente figlio non pu� modificare direttamente nessuna propriet� che gli � stata passata, ma pu� passare funzioni che possono modificare il valore. \textcolor{red}{(introdurre esempio)}
	\item \textbf{Virtual DOM}
	React crea una propria struttura dati in memoria, che calcola le differenze risultanti e aggiorna il DOM HTML risultante nella pagina in modo efficiente. In questo modo il programmatore pu� scrivere codice come se l'intera pagina � ricaricata ogni volta, mentre � la libreria React a stabilire quali siano i componenti che vadano sostituiti e quali no. 
	\item \textbf{JSX}
	I componenti React sono solitamente scritti in JSX, un'estensione JavaScript che permette di quotare facilmente HTML e utilizzarne la sintassi per i tag per renderizzare i sottocomponenti
\end{itemize}

\subsection{React Native}
React Native � una libreria open-source rilasciata nel 2015 sempre da Facebook, proponendosi come estensione del framework React per quanto riguarda le applicazioni mobile.
Le principali differenze con React sono dovute alla difficolt� maggiore di sviluppare un framework per le applicazioni mobile. 
Quando si sviluppa sul web, semplicemente si salvano i file modificati e si ricarica la pagina, cosa che non � possibile fare con le applicazioni mobile, perch� � necessaria una nuova compilazione per vedere i cambiamenti apportati.
Con React Native � possibile migliorare l'esperienza d'uso sulle piattaforme mobile rispetto al web. Come primo aspetto si pu� accedere ai componenti specifici dell'interfaccia utente, come le mappe, i picker e gli switch, anche se tuttavia possono essere reimplementati e modificati.


\section{Struttura dei file di mashup}

abc

\section{Rendering delle view}

abc

\section{Flusso di esecuzione}

abc

\section{Utilizzo dei dati}

abc

\subsection{Costruzione query GraphQL}

abc

\subsection{Gestione paginazione}

abc

\subsection{Gestione errori ed aggiornamenti}

abc

\section{Servizi di supporto}

abc