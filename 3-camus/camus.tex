Questo capitolo fornisce una panoramica sulle caratteristiche del progetto CAMUS, mettendo in evidenza le fondamenta sopra alle quali è stato sviluppato il framework. Nelle sezioni seguenti viene fornita inizialmente una panoramica del progetto e si illustrano gli obiettivi che si vogliono prendere in considerazione e per i quali viene proposta una soluzione. In seguito viene trattata la metodologia che integra alcuni precedenti lavori all'interno del progetto. Infine viene presentato un esempio di applicazione del progetto per risolvere un problema del mondo reale.

\section{Panoramica del progetto\label{sec:panoramica-progetto}}

CAMUS è l'acronimo di Context-Aware Mobile mashUpS e, come si può intuire dal nome, il principale obiettivo del progetto è quello di proporre un framework per la realizzazione di mashup mobile che sfruttino il contesto nel quale si trova l'utente al fine di proporgli informazioni che sono di suo interesse.

Ai giorni nostri si sente sempre più spesso parlare di \emph{Big Data}. Il termine di per sè è abbastanza fuorviante. La traduzione letterale \virgolette{grandi dati} o \virgolette{grossi dati} cattura solo l'aspetto di avere un'enorme quantità di informazioni a disposizione. L'altro aspetto, che è anche quello più importante, riguarda l'analisi di queste informazioni, che possono essere ricevute da diverse fonti e nei modi più disparati, al fine di comprendere come trattarle e capire cosa significano. Senza questa fase di \emph{comprensione}, avere a disposizione tutti questi dati porta al fenomeno del \emph{sovraccarico cognitivo}, o \emph{information overload}, cioè la difficoltà che una persona ha nel comprendere un problema e nel prendere una decisione quando ha a disposizione un'eccessiva quantità di informazioni. Questa enorme mole di dati permette sì di avere una conoscenza praticamente completa riguardo qualsiasi tematica, ma d'altro canto provoca anche dei problemi sulla scelta di \emph{quali} siano le informazioni effettivamente necessarie. 
 
 Il progetto CAMUS vuole proporsi come una soluzione user-friendly a questo problema e in particolare sfrutta due filoni di ricerca che a loro volta propongono una metodologia per attenuare questo problema, utilizzando approcci differenti. Questi due ambiti sono quelli relativi gli studi sul \emph{contesto} e sui \emph{mashup}. Viene utilizzato il \emph{contesto} per raccogliere informazioni sull'utente e sull'ambiente nel quale si trova, al fine di selezionare i servizi più idonei alla situazione e filtrare le informazioni che sono di maggior interesse. Oltre a questo aspetto, si vuole sfruttare la dinamicità di presentazione che caratterizza i \emph{mashup} per adattare la modalità nella quale queste informazioni vengono mostrate e la flessibilità di integrazione dei dati provenienti da diverse fonti.

In particolare, come precedentemente citato nella Sezione \ref{sec:mobile-mashup}, viene applicato il concetto dei mashup al mondo mobile: il risultato finale consiste in un'app per smartphone o tablet che adatta il proprio aspetto in base a delle regole di visualizzazione definite a priori. Questa caratteristica permette di ottenere un'estrema flessibilità, in quanto non è necessario rilasciare versioni diverse per ogni modifica nell'aspetto grafico. Viene lasciata massima libertà di variare l'interfaccia grafica in modo che si adatti a diverse situazioni di utilizzo.

CAMUS è strutturato in modo da garantire modularità e disaccoppiamento tra i vari componenti che ne fanno parte. I componenti principali del framework sono il \emph{server} e la \emph{mobile app}. Il \emph{server} è il cuore del sistema e orchestra la comunicazione tra gli altri componenti. Il suo compito primario è quello di fornire le informazioni idonee all'app mobile, a partire dal contesto che quest'ultima gli invia. La \emph{mobile app} viene utilizzata dall'utente finale e contiene il motore di rendering che interpreta le regole definite sul come mostrare i componenti e li disegna in modo da essere fruibili.
CAMUS prevede l'integrazione di alcune \emph{web app}, con lo scopo di permettere la realizzazione intuitiva della rappresentazione del contesto e dell'interfaccia grafica assunta dall'app. Le web app principali sono due: \emph{i)} la prima si propone di comporre il \emph{contesto}, che viene rappresentato tramite un modello idoneo alla visualizzazione grafica; \emph{ii)} la seconda riguarda la composizione degli elementi che formano l'interfaccia grafica della mobile app.

\section{Obiettivi del progetto\label{sec:obiettivi-progetto}}

L'obiettivo principale del progetto CAMUS è definire la metodologia con la quale è possibile creare app sfruttando i mashup ed il contesto, al fine di mostrare all'utente le informazioni che sono più pertinenti alla situazione in cui si trova.

Per la creazione del framework sono state rispettate le seguenti linee guida:

\begin{itemize}
	\item \textbf{Creazione visuale} Questo punto è relativo alla modalità di realizzazione delle app. Si vuole ideare un sistema che permetta la composizione dell'interfaccia grafica interamente in modo visuale, senza la necessità di scrivere codice
	\item \textbf{Contestualità} Si vuole sfruttare il contesto dell'utente per comprendere quali siano le sue esigenze e mostrargli così le informazioni maggiormente pertinenti alla situazione. In questo modo è possibile utilizzare anche diverse fonti per acquisire dati più precisi e variegati, che verranno filtrati adeguatamente in modo da mostrare le informazioni più rilevanti e nascondere quelle che porterebbero solamente a una maggior confusione
	\item \textbf{Semplicità} Ogni operazione deve essere facile, in modo che anche persone non pratiche di informatica possano creare delle app CAMUS. Qualsiasi attività di creazione e modifica viene dunque effettuata tramite un'interfaccia grafica, in modo da aiutare l'utente nella composizione dell'applicazione. Inoltre sono stati scelti dei modelli concettuali semplici, di facile comprensione ma sufficientemente potenti da creare schemi di una certa complessità
	\item \textbf{Automazione} Si vuole far ripetere all'utente il minor numero di azioni possibile; in quest'ottica vengono sfruttati i sensori disponibili sui dispositivi per acquisire in maniera automatica determinate informazioni, sgravando l'utente da questo compito
	\item \textbf{Uniformità dell'accesso ai dati} Il framework non prevede l'interfacciamento con un database per il recupero delle informazioni, bensì i dati vengono acquisiti tramite l'utilizzo dei \emph{servizi}. Questa soluzione rende più flessibile l'accesso ai dati, in quanto viene fornita un'interfaccia comune per servizi sia interni, cioè gestiti direttamente dal creatore dell'app, sia esterni, cioè mantenuti da terzi. Inoltre apre la strada all'utilizzo dei servizi di supporto, che forniscono dati complementari che arricchiscono le informazioni finali
	\item \textbf{Personalizzazione} Una delle principali caratteristiche del framework è la possibilità di modificare l'aspetto dell'app in modo intuitivo. Viene dunque agevolata la personalizzazione dell'interfaccia grafica non solo in base alla situazione di utilizzo ma anche tenendo conto del profilo dell'utente. Ogni utilizzatore dell'app può così avere una visualizzazione delle informazioni adatta alle sue esigenze, cosicché si possa concentrare esclusivamente sulla fruizione dei contenuti. Verrà dunque preferito uno schema flessibile che permetta la generazione dinamica delle schermate da mostrare all'utente e soprattutto verrà privilegiato un metodo semplice e visuale per la creazione di questi schemi
	\item \textbf{Universalità} CAMUS deve essere utilizzabile in ambiti diversi. Si vuole realizzare un modello che sia valido e in grado di essere applicato in contesti anche molto diversi tra loro. Per questo motivo il framework non viene costruito con elementi specifici di un determinato settore, proprio per evitare perdite di generalità
\end{itemize}

\section{Integrazione del contesto con i mashup\label{sec:integrazione-contesto-mashup}}

Come evidenziato nella Sezione \ref{sec:panoramica-progetto}, il progetto CAMUS consiste nell'unione delle ricerche effettuate nell'ambito della \emph{context-awareness} e dei \emph{mashup}. Il framework mira all'integrazione tra questi due filoni di ricerca, in modo da catturare le potenzialità di entrambi. I \emph{mashup} permettono di utilizzare diverse fonti per acquisire una maggior quantità di informazioni e ottenere dei dati che sono complementari, finalizzati all'arricchimento delle descrizioni degli elementi. Inoltre mettono a disposizione una logica di creazione dinamica delle interfacce grafiche: permettono di sfruttare la conoscenza acquisita dalle diverse fonti per visualizzare i dati che sono di maggior interesse. Il problema nasce quando si ha a che fare con una quantità enorme di informazioni. \upe sì un bene avere a disposizione una moltitudine di dati, ma esiste il rischio concreto che questa mole di dati generi unicamente confusione e renda difficoltoso il lavoro di ricerca delle informazioni che sono di reale interesse. Proprio per mitigare questo problema viene introdotto l'utilizzo del \emph{contesto}. Quest'ultimo permette di filtrare le informazioni che sono rilevanti in una particolare situazione. Vengono dunque nascosti o messi in secondo piano tutti i dati che non sono di alcun interesse per la situazione corrente, permettendo così all'utente di concentrarsi unicamente su un sottoinsieme dei risultati più attinenti al contesto nel quale si trova.

Rispetto al paradigma di composizione presentato nella Sezione \ref{sec:mobile-mashup}, in CAMUS vengono separate la attività di \emph{acquisizione dati} e \emph{creazione dell'interfaccia grafica}. Questo disaccoppiamento permette di modificare l'aspetto delle app in modo molto flessibile, in quanto non è necessaria a priori la conoscenza della fonte esatta dalla quale proverranno le informazioni. Viene così fornita un'estrema libertà al designer nella scelta di \emph{come} comporre l'interfaccia grafica, in modo che possa concentrarsi solo su questo aspetto e non si debba preoccupare di come i dati vengano integrati tra loro.

Un'altra parte del framework viene dunque dedicata all'integrazione dei dati. \upe in questa attività che entra in gioco il \emph{contesto}. Grazie alle informazioni sulla situazione nella quale si trova l'utente, è possibile selezionare \emph{quali} sono i servizi che forniscono i migliori risultati e, una volta acquisiti i dati, possono fornire le informazioni necessarie per assegnare un \emph{punteggio} ai vari elementi, in modo da portare in primo piano quelli più rilevanti. Quest'attività viene divisa in tre parti principali:

\begin{enumerate}
	\item \textbf{Selezione dei servizi} Vengono selezionati, grazie alle informazioni di contesto, i servizi che sono più idonei per la situazione corrente
	\item \textbf{Acquisizione dei dati} Vengono interrogati i servizi scelti per acquisire i dati
	\item \textbf{Integrazione delle informazioni} I dati ricevuti dai vari servizi vengono integrati in modo da formare un unico dataset. Quest'operazione prevede tre fasi: \emph{i)} trasformazione delle risposte in una rappresentazione comune; \emph{ii)} fusione degli elementi duplicati; \emph{iii)} assegnamento di un punteggio agli elementi, tenendo conto delle informazioni fornite dal contesto
\end{enumerate}

\section{Utilizzo dei servizi\label{sec:utilizzo-servizi}}

Uno dei punti chiave del framework, seguendo la filosofia dei mashup, è di non essere limitati ad una singola fonte per l'acquisizione dei dati, bensì si vuole essere in grado di recuperare le informazioni da diverse sorgenti. In questo modo è possibile ottenere diversi vantaggi, come l'acquisizione di una maggiore quantità di risultati, ottenere informazioni più complete, ecc. Presenta però anche degli svantaggi: avere molti dati a disposizione può essere dispersivo ed esiste il rischio di ottenere entità ridondanti. Per fornire un'elevata flessibilità, il framework prevede il disaccoppiamento della logica di accesso ai dati: per l'acquisizione delle informazioni viene fornita un'unica interfaccia, rappresentata dai \emph{servizi}. Questo metodo permette di avere una descrizione unica sul come recuperare i dati, indipendentemente dal fornitore che li mette a disposizione. L'utilizzo dei servizi porta i seguenti vantaggi:

\begin{itemize}
	\item \textbf{Varietà}	L'utilizzo di un'interfaccia generica per l'accesso ai dati favorisce l'acquisizione di informazioni da fonti diverse. In questo modo è possibile ottenere dei dataset più ampi
	\item \textbf{Integrabilità} Ogni servizio restituisce i dati secondo una propria rappresentazione. Avere un unico metodo di accesso ai dati permette anche di descrivere come queste informazioni vengano trattate una volta ricevute. Questo passaggio permette di uniformare le rappresentazioni fornite in ingresso, in modo tale che siano trasformate in una forma più idonea per le future elaborazioni
	\item \textbf{Completezza} Acquisendo informazioni da diverse fonti è possibile ottenere degli elementi duplicati. Questa caratteristica potrebbe sembrare un difetto mentre in realtà fornisce una risorsa molto importante: la possibilità di ottenere degli elementi più completi. Diverse fonti possono essere utilizzate per acquisire informazioni di entità diversa, in modo da generare un elemento descritto in maniera più precisa e che colga maggiori aspetti
\end{itemize}

Questa scelta permette inoltre di dividere l'acquisizione dei dati in due passaggi: \emph{i)} acquisizione delle informazioni principali; \emph{ii)} acquisizione di dati che vanno ad arricchire le informazioni principali. 

In particolare verrà utilizzato uno schema che permette di descrivere come accedere alle risorse messe a disposizione da un servizio e come interpretare le risposte ricevute. Questo formalismo possiede il vantaggio di essere utilizzabile per entrambe le casistiche presentate.

\section{Caso di studio: il turismo\label{sec:caso-studio-turismo}}

In questa sezione viene introdotto un caso di studio, che serve per validare il modello proposto per CAMUS. Utilizzare un esempio di applicazione del framework ad un caso reale può essere un utile indicatore al fine di supportare le scelte effettuate e di osservare se sono presenti alcune lacune nella modellazione.

CAMUS ha come obiettivo quello di essere universale, cioè utilizzabile in ambiti anche molto diversi tra loro. Come caso di studio è stato dunque necessario pensare a un ambito che consentisse di sfruttare al meglio le capacità di adattarsi alle varie situazioni nelle quali si può trovare l'utilizzatore dell'app.

\upe stato scelto il caso di studio relativo al settore \emph{turistico}, proprio per via della sua dinamicità. In particolare si è preso in considerazione il caso di un'agenzia di viaggi che deve proporre ai propri clienti dei pacchetti di viaggio.

L'ambito turistico calza a pennello per gli intenti di CAMUS; un viaggio incomincia dalla sua pianificazione: prima di partire, il turista si informa sulla destinazione, quali hotel sono disponibili, con quali mezzi di trasporto è più conveniente raggiungere la destinazione, ecc. In questo frangente CAMUS si propone come \virgolette{suggeritore} di esperienze di viaggio: permette all'utente di non concentrarsi sul \emph{dove} vuole andare, ma sul \emph{cosa} vuole fare. Gli viene data la possibilità di scegliere l'esperienza di viaggio che vuole intraprendere e proporgli così i pacchetti che corrispondono alle sue scelte.

Le potenzialità di CAMUS non terminano una volta che il viaggiatore sceglie la sua meta: l'app gli servirà \emph{durante} il viaggio per informarsi sulle attività da svolgere, le escursioni disponibili, i ristoranti nei quali cenare, ecc. Per esempio, se un utente vuole provare un ristorante tipico della zona, gli basterà aprire l'app CAMUS e selezionare che è interessato nella ricerca dei ristoranti che propongono dei menù \emph{tipici} del luogo. Gli verrà mostrato un elenco dei ristoranti che si trovano nei suoi paraggi che rispettano i criteri definiti. Inoltre gli verranno fornite indicazioni su come raggiungere il ristorante, sfruttando, se specificato nella richiesta, i mezzi pubblici presenti nella destinazione scelta.

Le due figure principali in questo caso di studio sono due: l'\emph{agente di viaggio} e il \emph{turista}. L'\emph{agente} ha il compito di organizzare tutto il viaggio, gestire quindi le prenotazioni, gli spostamenti principali e i soggiorni per il viaggiatore, mentre il \emph{turista} è colui che fisicamente compie il viaggio, che può essere quindi considerato come l'utente finale.

All'agente vengono inoltre demandati i compiti di personalizzazione dell'app: questa fase avviene quando il cliente si presenta nell'agenzia di viaggio. A quest'ultimo vengono prima di tutto fatte alcune domande per conoscere il suo profilo, in modo da adattare le scelte che gli verranno mostrate dall'app. In questo modo è possibile evitare che al cliente vengano proposte troppe scelte che non siano di suo interesse. Inoltre vengono fissate alcune opzioni che tendono a non cambiare nel corso del viaggio; per esempio se l'utente viaggia con un animale domestico questa scelta sarà sempre la stessa durante tutto il viaggio e non gli verrà chiesta di nuovo.%\footnote{CAMUS si schiera contro l'abbandono degli animali: \url{http://www.emergenza24.org/salvaunamico/}}

L'ulteriore compito che può svolgere l'agente è quello di personalizzare l'aspetto grafico dell'app qualora ritenga che sia necessario. Con CAMUS vengono proposti alcuni template predefiniti per ogni categoria e viene lasciata libera scelta all'agente se utilizzarli così come sono o modificarli. Una delle motivazioni nella scelta di modificare l'aspetto riguarda l'aggiunta di alcuni servizi di supporto. Per esempio, se l'agente è consapevole che nel luogo in cui vuole andare il cliente le condizioni meteorologiche sono molto variabili e nella versione predefinita dell'app non sono presenti informazioni meteo, può aggiungere questa indicazione in modo da fornire al turista un'informazione in più per gestire al meglio la sua vacanza.

Come si può notare da questo breve esempio, l'utilizzo di CAMUS in un'agenzia viaggi permette di fornire un servizio più coinvolgente ai propri clienti. La realizzazione di app personalizzate appositamente per un turista permette di effettuare scelte migliori e di proprio interesse. Inoltre estende la capacità dell'agenzia di fornire un'assistenza al proprio cliente anche durante il viaggio: si può affermare che un'app CAMUS assume il ruolo di \emph{assistente di viaggio} e guida il turista durante tutte le fasi della sua esperienza di viaggio.