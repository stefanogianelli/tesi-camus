In questo capitolo ...

\section{Panoramica del progetto}

\textcolor{red}{Introduzione al progetto\\
	Spiegare le due anime, senza dire il perch�\\
	Dare un'idea di come sono strutturati i vari ambienti (server, web app e app mobile)}

\section{Obiettivi del progetto\label{sec:obiettivi-progetto}}

\textcolor{red}{Descrivere gli obiettivi del progetto\\
	Creazione di app in modo semplice e visuale\\
	sfruttamento del contesto per migliorare le ricerche\\
	strumenti del contesto molto semplici per agevolare la modifica}

\section{Perch� integrare contesto e mashup?}

\textcolor{red}{"Limiti" delle due aree\\
	dire come mai sono complementari, come possono essere unite e i vantaggi che porta\\
	specificare la divisione tra gestione dati, che viene demandata soprattutto al contesto, e gestione interfacce, che viene data ai mashup\\
	dire anche perch� si � tolta la parte di gestione dati dai mashup}

\section{Come mai vengono utilizzati i servizi?}

\textcolor{red}{Spiegare l'utilit� dei servizi come disaccoppiamento da un DB specifico\\
	I servizi possono essere sia interni che esterni\\
	accennare ai servizi di supporto}

\section{Caso di studio del turismo\label{sec:caso-studio-turismo}}

Al fine di pensare a come specificare il progetto CAMUS, � stato necessario introdurre un caso di studio che fosse sufficientemente attinente al nostro progetto.
La scelta � caduta sull'agenzia viaggi, un ambito di tipo turistico, perch� pu� nascere la necessit� di avere un sistema di mashup che fornisca dati di diversa natura al viaggiatore, a seconda del contesto nel quale si trova.
Le due figure principali in questo caso di studio sono l'agente di viaggio e il viaggiatore. L'agente ha il compito di organizzare tutto il tour, con tutte le prenotazioni, spostamenti principali e soggiorni per il viaggiatore, mentre il viaggiatore � colui che fisicamente compie il viaggio, quindi lo possiamo considerare come l'utente finale.
Quest'ultimo, oltre ad essere gi� in possesso delle prenotazioni per mezzi di trasporto, pernottamento ed eventuali extra di soggiorno, pu� avere la necessit� di ottenere informazioni aggiuntive.
Con la grande quantit� di informazioni e risorse presenti sul Web, non � sempre vero che l'utente abbia una maggiore conoscenza, anzi, alcune volte, si ottiene l'effetto contrario, in quanto non � in grado di filtrare le informazioni per le sue esigenze. Data la grande diffusione di dispositivi mobili, si � scelto di introdurre questo sistema mobile context-aware e in questo modo poter offrire questa nuova esperienza d'uso per il cliente. Per mezzo della web app del sistema, l'agente di viaggio pu� configurare per il viaggiatore un'applicazione personalizzata, la quale � in grado di adattarsi alle diverse condizioni di utilizzo dell'utente. La dinamicit� dell'applicazione � dettata dal cambiamento dei servizi invocati dal sistema a seconda delle variazioni nel contesto.
Ad esempio, a Milano nell'ora di cena, se l'utente seleziona come argomento di interesse i ristoranti ed � senza automobile, saranno invocati servizi adatti ad offrire questo tipo di risultati all'utente, con i dovuti mezzi di trasporto locale. Se il viaggiatore si spostasse verso Torino, l'applicazione invocherebbe servizi diversi per fornire dei dati migliori all'utente.
Questa definizione ed associazione dei dati viene fatta dall'agente di viaggio,il quale si suppone che abbia una certa esperienza nel saper scegliere quali servizi sono pi� attinenti a un certo elemento dell'albero di contesto, da cui il sistema, con operazioni spiegate nel Capitolo \ref{ch:metodologia}, sceglie quali servizi invocare dinamicamente.
