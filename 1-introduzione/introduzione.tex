\textcolor{red}{Parlare dei big data, in particolare dei problemi con la quantità di dati e necessità di mettere ordine, dare significato\\
	prendere spunto da articolo and what can context do for data\\
	trovare grafici\\
	obiettivi della tesi}

Al giorno d'oggi il Web permette di acquisire una grandissima quantità di informazioni, grazie ai progressi in ambito tecnologico. 

\textcolor{red}{Grafico dati nel web} 

Questa è la \emph{Information Age}, cioè il fenomeno dove la conoscenza è alla base della società e la tecnologia influenza il modo in cui operano le industrie e gli operatori dei servizi, permettendogli di agire in modo più efficiente e conveniente. Tutte queste informazioni permettono a chiunque di esplorarle secondo le proprie esigenze. 

\textcolor{red}{parlare dei pro nell'acquisire informazioni da diverse fonti}

Il problema che ora si pone è quello che i ricercatori chiamano \virgolette{dare senso ai dati}, i quali danno un valore aggiunto solo utilizzando gli algoritmi appropriati. Questa grande quantità di dati risulterebbe inutile se non si fosse in grado di sfruttarla. Infatti il valore aggiunto è proprio la possibilità di analizzarli, estrarre informazioni utili e ricavarne conoscenza. 

%Inizio parte del contesto
Questo aumento della quantità di informazioni, se non propriamente controllato, può provocare un ammasso di dati che porta soltanto confusione piuttosto che fornire informazioni utili, riducendo i benefici che invece si potrebbero ricavare da tutte queste informazioni. Tuttavia, distinguere le informazioni rilevanti da quelle che non lo sono non è un compito semplice; alcune informazioni potrebbero essere trattate in maniera differente, anche per lo stesso utente, che in diverse situazioni o posti ha bisogno di informazioni differenti. 

\textcolor{red}{Descrivere un po' il contesto}

%Resta aperto il problema di come visualizzare le operazioni e come ricercare le info in modo semplice (Fornire interfaccia per permettere agli utenti di cercare le informazioni)

Il contesto permette dunque di acquisire informazioni rilevanti per l'utente, ma non definisce nessuna regola su come verranno visualizzati all'utente finale. Con la diffusione di dispositivi mobili sempre più sofisticati si è resa maggiormente necessaria un'esperienza utente semplice che lo guidi durante le sue attività 

\textcolor{red}{Ruolo dell'user experience}

Vengono dunque in soccorso i \emph{Mashup}, che permettono la realizzazione di interfacce grafiche estremamente dinamiche. In questo modo può essere facilmente rappresentato il contesto e definire un aspetto idoneo per le varie categorie di risultati. \textcolor{red}{Ogni categoria di informazioni ha le sue peculiarità e avere schemi diversi per categoria di dati ha il suo perché}



\section{Struttura della tesi\label{sec:struttura-tesi}}




\textcolor{red}{descrivere struttura della tesi}