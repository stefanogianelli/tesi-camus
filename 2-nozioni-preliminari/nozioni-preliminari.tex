
\section{Context-Awareness\label{sec:context-awareness}}

\textcolor{red}{STEFANO\\Informazioni storiche sulle ricerche relative al contesto} \cite{DBLP:journals/sigmod/BolchiniCQST07} \cite{baldauf2007survey}

\subsection{Context Dimension Model\label{sec:contex-dimension-model}}

\textcolor{red}{STEFANO\\Definizione del CDT}

In questa sezione verr� presentato il \emph{Context Dimension Tree} \cite{DBLP:journals/is/BolchiniQT13}, che � il modello di rappresentazione del contesto che verr� utilizzato in CAMUS.

\section{Mobile Mashup\label{sec:mobile-mashup}}

\textcolor{red}{VALERIO\\Informazioni generali sui mashup (non solo mobile)\\
Modello Visuale utilizzato\\
Sottosezione relativa a React Native (spiegare profonda integrazione con la piattaforma di riferimento e confronto con le webview)}

\section{Web Services\label{sec:web-services}}

\textcolor{red}{STEFANO\\Introduzione su cosa sono i servizi web\\
Caratteristiche dei servizi web (free, pagamento, paradigmi, ...)\\
Classificazione dei servizi esistenti (SOAP, REST, ...) e modi di interrogazione\\
Descrizione GraphQL, problemi che vuole risolvere e paragone con REST}

\section{Stato dell'arte\label{sec:stato-arte}}

\textcolor{red}{Descrizione dei principali "competitor" (IFTTT, Atooma, Swagger, Appery.io, Kimono, AdAPT, Yahoo Pipes, JackBe Presto, Mashart.org, Peudom)}